%!TEX root = ../Diploma.tex

\clearpage
\begin{abstract}
\makestat{Магистерская диссертация}

СОЦИАЛЬНЫЕ СЕТИ, СПАМ, ЗАДАЧА КЛАССИФИКАЦИИ, МАШИННОЕ ОБУЧЕНИЕ

\textbf{Объект исследования}~--- применение алгоритмов машинного обучения для классификации спама в социальных сетях.

\textbf{Цель работы}~--- исследование методов определения спама в социальных сетях, сравнение подходов для решения задачи классификации спама, построение модели распознавания спама в социальной сети Twitter на основе алгоритмов машинного обучения.

\textbf{Методы исследования}~--- наивный байесовский классификатор, метод \textit{k} ближайших соседей, метод опорных векторов (SVM), решающие деревья, случайные леса.

\textbf{Результатом} является предложенный подход для построения классификатора социального спама, не требующий наличия историчных признаков пользователя социальной сети.

\textbf{Область применения}~--- системы спамообороны.
\end{abstract}


\pagebreak


\begin{abstract:en}
\makestaten{Master thesis}

SOCIAL NETWORKS, SPAM DETECTION, CLASSIFICATION PROBLEM, MACHINE LEARNING

\textbf{Object of research}~--- Machine learning algorithms application in social spam classification problem.

\textbf{Purpose}~--- studying methods of social spam detection, comparing approaches of solving the problem of spam classification, building a model for Twitter spam detection based on machine learning algorithms.

\textbf{Research methods}~--- naive Bayes classifier, \textit{k}-nearest neighbors algorithm, support vector machines (SVM), Decision trees, Random forest.

\textbf{Results} is the social spam classifier that does not require user historical features.

The results can be applied in antispam systems.
\end{abstract:en}
