%!TEX root = ../Diploma.tex

\clearpage
\begin{abstract:ru}
\makestat{Магистерская диссертация}

СОЦИАЛЬНЫЕ СЕТИ, СПАМ, ЗАДАЧА КЛАССИФИКАЦИИ, МАШИННОЕ ОБУЧЕНИЕ

\textbf{Объект исследования}~--- применение алгоритмов машинного обучения для классификации спама в социальных сетях.

\textbf{Цель работы}~--- исследование методов определения спама в социальных сетях, сравнение подходов для решения задачи классификации спама, построение модели распознавания спама в социальной сети Twitter на основе алгоритмов машинного обучения.

\textbf{Методы исследования}~--- наивный байесовский классификатор, метод \textit{k} ближайших соседей, метод опорных векторов (SVM), решающие деревья, случайные леса.

\textbf{Результатом} является предложенный подход для построения классификатора социального спама, не требующий наличия историчных признаков пользователя социальной сети.

\textbf{Область применения}~--- системы спамообороны.
\end{abstract:ru}

\pagebreak

\begin{abstract:by}
\makestat{Магістарская дысертацыя}

САЦЫЯЛЬНЫЯ СЕТКІ, СПАМ, ЗАДАЧА КЛАСІФІКАЦЫі, МАШЫННАЕ НАВУЧАННЕ

\textbf{Аб’ект даследавання}~--- прымяненне алгарытмаў машыннага навучання для класіфікацыі спаму ў сацыяльных сетках..

\textbf{Мэта працы}~--- даследаванне метадаў вызначэння спаму ў сацыяльных сетках, параўнанне падыходаў для вырашэння задачы класіфікацыі спаму, пабудова мадэлі распазнання спаму ў сацыяльнай сетцы Twitter на аснове алгарытмаў машыннага навучання.

\textbf{Метады даследавання}~--- наіўны байесовский класіфікатар, метад \textit{k} бліжэйшых суседзяў, метад апорных вектараў (SVM), вырашальныя дрэвы, выпадковыя леса.

\textbf{Вынікам} з'яўляецца прапанаваны падыход для пабудовы класіфікатара сацыяльнага спаму, які не патрабуе наяўнасці гістарычных прыкмет карыстальніка сацыяльнай сеткі.

\textbf{Вобласць выкарыстання}~--- сістэмы самаабароны.
\end{abstract:by}

\pagebreak


\begin{abstract:en}
\makestaten{Master thesis}

SOCIAL NETWORKS, SPAM DETECTION, CLASSIFICATION PROBLEM, MACHINE LEARNING

\textbf{Object of research:} Application of machine learning algorithms in social spam classification problem.

\textbf{Purpose}~--- studying methods of detection spam in social networks, comparing approaches to solving the problem of spam classification, building a model for spam recognition in the social network Twitter based on machine learning algorithms.

\textbf{Research methods}~--- naive Bayes classifier, \textit{k}-nearest neighbors algorithm, support vector machines (SVM), Decision trees, Random forest.

\textbf{Results} is the proposed approach for building a social spam classifier that does not require the historical features of the user of the social network.

The results can be applied in antispam systems.
\end{abstract:en}
