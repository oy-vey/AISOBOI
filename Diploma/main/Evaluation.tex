%!TEX root = ../Diploma.tex


\begin{section}{Проведенные эксперименты}

\begin{subsection}{Выбор классификатора}

В качестве метрик качества классификаторов были выбраны полнота (R), точность (P) и F-1 статистика. Метрика полноты представляет из себя отношение числа сообщений, корректно классифицированных как спам (True Positives) и общего числа спамовых сообщений (True Positives + False Negatives):

\begin{equation}
  R = \frac{True Positives}{True Positives + False Negatives}
\end{equation}

Точность - это отношение числа сообщений, корректно классифицированных как спам (True Positives) и общего числа  сообщений, классифицированных как спам (True Positives + False Positives):

\begin{equation}
  P = \frac{True Positives}{True Positives + False Positives}
\end{equation}

F-1 статистика может быть проинтерпретирована как гармоническое среднее между метриками точности и полноты:

\begin{equation}
  F1 = \frac{2 \times P \times R}{P + R}
\end{equation}



Для обучения из имеющегося набора данных случайным образом был использован







\begin{table}[H]
\centering
\resizebox{\textwidth}{!}{\begin{tabular}{|l|c|c|c|}
\hline
\textbf{Классификатор} & \textbf{Точность} & \textbf{Полнота} & \textbf{F-1}  \\
\hline
kNN & 86\% & 86\% & 86\% \\
\hline
Решающее дерево & 89\% & 89\% & 89\% \\
\hline
Наивный байесовский классификатор & 76\% & 76\% & 76\% \\
\hline
\textbf{Случайные леса} & \textbf{93\%} & \textbf{93\%} & \textbf{93\%} \\
\hline
SVM & 86\% & 86\% & 86\%  \\
\hline
\end{tabular}}

\caption{Сравнение показателей классификатора}
\label{tab:results}
\end{table}




\begin{table}[H]
\centering
\resizebox{\textwidth}{!}{\begin{tabular}{|l|c|}
\hline
\textbf{Группа признаков} & \textbf{Время на вычисление (сек.) для 1000 твитов} \\
\hline
Пользовательские признаки & 0.0057 \\
\hline
Признаки контента & 1.0448 \\
\hline
\end{tabular}}

\caption{Время вычисления признаков}
\label{tab:comptime}
\end{table}


\begin{table}[H]
\centering
\resizebox{\textwidth}{!}{\begin{tabular}{|l|c|c|}
\hline
\textbf{К} & \textbf{Точность} & \textbf{Кросс Валидация} \\
\hline
k =  & 0.0057 & \\
\hline
k =  & 1.0448 & \\
\hline
k =  & 0.0057 & \\
\hline
k =  & 1.0448 & \\
\hline
k =  & 0.0057 & \\
\hline
k =  & 1.0448 & \\
\hline
k =  & 0.0057 & \\
\hline
k =  & 1.0448 & \\
\hline
k =  & 0.0057 & \\
\hline
k =  & 1.0448 & \\
\hline
k = & 0.0057 & \\
\hline
k = & 1.0448 & \\
\hline
\end{tabular}}

\caption{Время вычисления признаков}
\label{tab:comptime}
\end{table}





\end{subsection}


\end{section}
