%!TEX root = ../Diploma.tex

\clearpage
\sectionc*{ВВЕДЕНИЕ}{
\addcontentsline{toc}{section}{ВВЕДЕНИЕ}
Проблема спама в социальных сетях, иначе говоря, социального спама, наносит все больший ущерб для таких компаний, как Facebook, Twitter, Pinterest и т.д. Согласно исследованию компании Nexgate, специализирующейся на безопасности пользователей социальных сетей, вышеупомянутые платформы столкнулись с 355\% ростом объёма социального спама в первой половине 2013 года \cite{Nexgate}.

Социальный спам влечет за собой последствия самого различного рода, и как следствие, существует несколько определений понятия социального спама. Одна из самых популярных социальных платформ Twitter дает собственное определение спама и предоставляет несколько возможных способов сообщать о спамовых активностях. Примером такого сообщения может служить твит  «@spam @username», где @username – никнейм потенциального спамера. При этом будучи коммерческой платформой, Twitter достаточно лояльно относится к сообщениям рекламного характера, при условии, что они не нарушают правила и политику использования социальной сети. В последние годы вышло в свет огромное количество работ и исследований, посвященных выявлению различных скрытых трендов и тенденций в самых разнообразных областях, начиная от финансовых рынков и заканчивая общественно-политической сферой. Огромная часть этих исследований строилась на основе открытых данных пользователей Twitter, что подтверждает огромную значимость этой платформы, как для академической, так и для индустриальной части общества. Потребность в выявлении значимой информации в шумовом поле Twitter является одной из ключевых задач подобных исследований, и в этом смысле, огромные обьемы социального спама очень серьезно усложняют эту задачу.

В опубликованных ранее работах, задача автоматизации поиска спама рассматривалась с 2-х точек зрения. Первая – подход, основанный на классификации конкретного пользователя, как спамера или не спамера.  Этот подход наиболее популярен в большинстве работ на данный момент (см. \cite{Wang} \cite{Benevenuto} \cite{McCord} \cite{Lee} \cite{Yang} \cite{Ferrara}) и требует

Целью работы является исследование и разработка методов автоматического распознавания спам-сообщений в социальной сети Twitter. При написании работы были поставлены следующие задачи:
\begin{enumerate}
\item Исследование политики, применяемой социальной сетью Twitter относительно спама, а также техник, используемых спамерами.
\item Проведение анализа существующих методов распознавания спама, а также оценки их эффективности.
\item Разработка методов автоматического распознавания спама.
\item Реализация разработанных методов и проведение экспериментальной оценки результатов их работы.
\end{enumerate}
}
