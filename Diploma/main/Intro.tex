%!TEX root = ../Diploma.tex

\clearpage
\sectionc*{ВВЕДЕНИЕ}{
\addcontentsline{toc}{section}{ВВЕДЕНИЕ}
Проблема спама в социальных сетях, иначе говоря, социального спама, наносит все больший ущерб для таких компаний, как Facebook, Twitter, Pinterest и т.д. Согласно исследованию компании Nexgate, специализирующейся на безопасности пользователей социальных сетей, вышеупомянутые платформы столкнулись с 355\% ростом объёма социального спама в первой половине 2013 года \cite{Nexgate}.

Социальный спам влечет за собой последствия самого различного рода, и как результат, существует несколько определений понятия <<социальный спам>>. Одна из самых популярных социальных платформ Twitter дает собственное определение спаму и предоставляет несколько различных способов сообщать о спамовых активностях. Примером такого сообщения может служить твит  «@spam @username», где @username – никнейм потенциального спамера. При этом будучи коммерческой платформой, Twitter достаточно лояльно относится к сообщениям рекламного характера, при условии, что они не нарушают правила и политику использования социальной сети. В последние годы вышло в свет огромное количество работ и исследований, посвященных выявлению различных скрытых трендов и тенденций в самых разнообразных областях, начиная от финансовых рынков и заканчивая общественно-политической сферой, огромная часть которых строилась на основе открытых данных пользователей Twitter, что подтверждает огромную значимость этой платформы, как для академической, так и для индустриальной части общества. Потребность в выявлении значимой информации в шумовом поле Twitter является одной из ключевых задач подобных исследований, и в этом смысле, огромные обьемы социального спама очень серьезно усложняют эту задачу.

Существующие техники и методы классификации социального спама в основной своей массе используют обширный обьем исторических данных о пользователе.
В данной работе была предпринята попытка создания классификатора спама на основе не столь большого набора необходимых данных, а лишь той информации, которую можно извлечь непосредственно из информации об авторе твита в момент его публикации. Значимым плюсом подобного классификатора может служить его способность своевременно определять спамовых пользователей и, как следствие, потенциальное применение в режиме реального времени.
Используя классифицированный вручную набор данных твитов, среди которых находились спамовые твиты, была предпринята попытка использования пяти алгоритмов классификации на основе четырех различных групп признаков.

Целью работы является исследование и разработка методов автоматического распознавания спам-сообщений в социальной сети Twitter. При написании работы были поставлены следующие задачи:
\begin{enumerate}
\item Исследование политики, применяемой социальной сетью Twitter относительно спама, а также техник, используемых спамерами.
\item Проведение анализа существующих методов распознавания спама, а также оценки их эффективности.
\item Разработка методов автоматического распознавания спама.
\item Реализация разработанных методов и проведение экспериментальной оценки результатов их работы.
\end{enumerate}
}

Структура данной работы следующая. Глава 1 знакомит читателя с основной характеристикой социальной сети Twitter, а также с существующей на данный момент политикой этой компании относительно спама.
В главе 2 приведен обзор и анализ ключевых исследований в сфере классификации социального спама. Глава 3 посвящена описанию техник и приемов, которые я использовал для своего классификатора.
В главе 4 приведены результаты попытки построения классификатора спама и оценка его качества.
