%!TEX root = ../Diploma.tex

\clearpage
\sectionc*{ЗАКЛЮЧЕНИЕ}{
\addcontentsline{toc}{section}{ЗАКЛЮЧЕНИЕ}
    В данной работе были рассмотрены различные подходы, которые применяются для построения систем определения социального спама в т.ч. с ограничением на число признаков. В результате работы были исследованы 5 алгоритмов машинного обучения, а именно наивный байесовский классификатор, метод \textit{k} ближайших соседей, метод опорных векторов, решающее дерево и случайный лес. Лучший результат продемонстрировал алгоритм Random Forest - $93$\%.

    Наиболее высокий известный мне результат в данной задаче составляет $98$\%, однако в нем отсутствуют какие-либо ограничения на историчность признаков, посему достигнутый результат можно считать довольно неплохим.

    В силу  отсутствия сьерьзной вычислительной нагрузки для извлечения использованных в работе признаков из твитов пользователей данный подход привлекателен своей потенциальной примененимостью в режиме реального времени.

    Весь исходный код, написанный в процессе работы над классификатором, можно найти в открытом Git-репозитории\footnote{https://github.com/oy-vey/TwitterSpamClassifier}.


}
